\section{Compliance Levels}
\label{section:complianceLevels}
Although SCJ was developed with simplification in mind, the complexity across different types of applications varies. Likewise, the complexity of a SCJ application, affects the certification process. In order to ease certification, SCJ defines three levels for this purpose, referred to as \textit{compliance levels}. The levels are denoted \textit{level~0}, \textit{level~1} and \textit{level~2}, where simple applications may conform to level~0 and complex ones to level~2. It should be noted that compliance levels are \textit{not} related to the criticality levels from the previously mentioned DO-178B standard. In the following, the different levels will be elaborated.

\subsection{Level 0}
A level~0 application executes under a model similar to the cyclic executive model described in Section \ref{subsec:cyclicExecutive}. Each mission within the application is basically broken down into fixed size computation parts and run in sequence. It is up to the developer to create the schedule, which entails creating the frames, distributing the computation parts (implemented as event handlers) amongst the frames, and setting the frame durations. Figure \ref{img:level0.pdf} shows a graphical example of an execution schedule consisting of four event handlers that are confined to a single mission. The schedule consists of three frames (the minor cycles) containing, one, two and one event handler(s) respectively. Notice that each event handler is assigned a unique priority, however, when executing the application as a level~0 application, the priorities are disregarded. Should the application run under a higher compliance level such as 1 or 2, the priorities are taken into account.

\img{level0.pdf}{0.8}{A level 0 application running under the cyclic executive scheduling model.}

In the cyclic executive approach, event handlers run in sequence within their respective frames and thereby also till completion --- i.e. no preemption occurs. Execution thus happens as if only a single thread exists. Although this has the benefit of being able to omit synchronisation mechanisms, the use of synchronisation is advised, should the application run at a higher compliance level. The use of the suspending method \code{Object.wait} and appertaining methods \code{Object.notify} and \code{Object.notifyAll} are not allowed in a level~0 application. Likewise, only synchronised methods may be used --- \code{synchronized} statements are not allowed.

\subsection{Level 1}
For level~1 applications and upwards, the scheduling scheme is changed to fixed-priority scheduling, described in Section \ref{sub:fps}, with priority ceiling emulation enabled. In addition, aperiodic event handlers can be used and all event handlers are explicitly bound to a thread each in the underlying platform. Figure \ref{img:level1.pdf} shows a graphical example of a typical level~1 application having a single mission with three schedulable objects, two periodic event handlers and a single aperiodic event handler. The handlers are interleaved as time progresses based on their priority and with preemption taking place during the first release of SO 1, which is caused the release of SO 2 due to its higher priority. Like level~0, the same restrictions regarding \code{Object.wait} and synchronised methods apply for level~1.

\img{level1.pdf}{0.8}{A level~1 application running under a fixed-priority scheduler.}

\subsection{Level 2}
The last level allows the creation of applications that require more dynamics. This includes support for both periodic and aperiodic event handlers as well as no-heap real-time threads. The most significant feature of a level~2 application is the possibility of creating nested missions as shown in Figure \ref{img:level2.pdf}.

\img{level2.pdf}{0.8}{A level 2 application running under fixed-priority scheduling with a nested mission.}

Although the expressiveness and potential seems great for applications in compliance level~2, we have been unable to find any SCJ implementations supporting level~2. Combined with the expectation that level~2 applications are more comprehensive and expensive to certify compared to the lesser levels, it is unlikely that (many) safety-critical applications will be developed as level~2.
