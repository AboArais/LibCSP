\section{The Programming Model}
\label{section:programmingmodel}
The programming model of SCJ can be considered being much more strict than its predecessor, RTSJ, which has been criticised for being too complex and difficult to certify under rigid standards such as the DO-178B level A - something that was also mentioned in Section \ref{sub:brief_history_of_java_for_safety_critical_systems}. This simplification naturally poses restrictions on the developer in terms of how code must be structured within a SCJ application, but fortunately eases the whole certification process - a topic that will be described later in this Chapter. Furthermore, by forcing the developers to adhere to a general programming model as well as having special annotations for vendor-supplied third-party tools that are able to analyse correctness properties of the source code depending on the different use contexts, aids in simplifying the certification. This Section will describe the programming model in SCJ, starting with the so-called \textit{missions}, and the rationale behind some of the decisions and concepts.


\subsection{Missions}
As a requirement and as the back-bone of every SCJ application, an application must consist of at least one \code{mission}. Furthermore \code{missions} can be regarded as entities encapsulating specific responsibilities of the overall application. This is much in accordance with the Object-Oriented Programming (OOP) principles of dividing code into classes having specific purposes. The granularity of encapsulation and thereby separation of concern is, however, often broader in \code{missions} as each mission can take the responsibilities of several tasks that are confined to a specific domain. As an example, consider the software for an unmanned moon rover taking earth samples on Mars. The rover needs to perform several vital functions in order to complete its objectives. The functionality can roughly be divided into:

\begin{itemize}
	\item Detach from charge station
	\item Navigate the environment to specific locations
	\item Align mounted solar panels to optimal positions throughout operation to prolong battery life
	\item Obtain earth samples
	\item Communicate results to outside world
	\item Drive back and attach to charge station
\end{itemize}

The above functionality could be represented in five different missions with coordination happening in between. Each mission would in turn contain a number of \textit{schedulable objects} (threads) refereed to as \textit{event handlers}. These will be described in the next section. "The Expert Group" behind the specification decided to include the mission concept as a solution to the problem of some modern safety-critical monolithic applications consisting of millions of lines of code making them very complex to maintain and certify. Also by dividing the application into one or more missions, each mission could be developed independently and in isolation of the others.  

To create a \code{mission} class in SCJ it must extend the abstract \code{Mission} as its base class and override the two abstract methods, \code{initialize} and \code{missionMemorySize}. In \code{initialize}, the mission context is initialized such as creating the \textit{event handlers} (tasks) and other objects like shared data structures. In \code{missionMemorySize}, the maximum memory space for the \code{mission} is defined (in bytes). Returning back to the rover example, a possible mission for the solar panel functionality, could look like that seen in Listing \ref{missionexample}. The reason for explicitly stating the memory consumption is due to the absence of automatic garbage collection. Furthermore as memory in real-time systems often is a limited resource, this number should be assigned with care by analysing the contained code.

\lstinputlisting[label=missionexample,caption=Example of a mission.]
{Code/mission.java}

\subsubsection{Mission Life Cycle}
The concept of a \textit{mission life cycle} covers the different phases a mission can be situated in throughout its "life time". Figure \ref{img:scj_mission_lifecyle.pdf} shows the various phases of a mission.

\img{scj_mission_lifecyle.pdf}{0.6}{The mission life cycle.}

Once the application starts to execute, a single mission is selected from all defined missions, by a \textit{mission sequencer}, that oversees the execution of each \code{mission}. Next the \code{mission} is initialised which involves calling the mission's \code{initialize} method (see Listing \ref{missionexample}) that amongst other things registers the defined \code{schedulable objects}. These are internally represented in the class, \code{ManagedScheduable}. 

Upon the transition from the \textit{initialisation phase} to the \textit{execution phase}, the \code{schedulable objects} that were defined and registered during the initialization phase, are scheduled by the underlying fixed-priority scheduler. Note that \code{schedulable objects} residing in other missions can not be scheduled at this time, but only those defined in the currently selected \code{mission}. 

Before terminating the currently executing mission and switching to the next, i.e. after every \code{scheduable object} within the mission have terminated, a \textit{clean up phase} occurs in which mission related clean up code can be executed. In order to do so, the following method must be overridden from \code{Mission:}

\begin{quotation}
	\code{protected void cleanUp();}
\end{quotation}

\subsection{Handlers}
\label{subsection:handlers}
A mission in itself does not do much apart from allocating shared data structures. The functional behaviour of a mission happens through the previously mentioned \code{schedulable objects}, which are basically implemented as threads that operates as event handlers. This will be elaborated in further detail in Section \ref{section:schedulingConcurrency}. The event handlers can be either periodic or aperiodic (sporadic tasks are not supported in SCJ). In the \code{initialize} method of the solar panel \code{mission} for the Mars rover in Listing \ref{missionexample}, a periodic event handler could be instantiated and registered.

\lstinputlisting[label=handlerexample,caption=Example of a handler.]
{Code/handler.java}

The supplied arguments denote its priority (\code{PriorityParamenters}), release time and frequency (\code{PeriodicParameters}) and various storage parameters. The handler class itself, being its a periodic event handler, must extend the class \code{PeriodicEventHandler}. Aperiodic event handlers similarly extends \code{AperiodicEventhandler}. As a minimum requirement, the class must have a constructor accepting the above mentioned parameters that will be passed to its immediate base class, but also override another contravariant method called \code{handleAsyncEvent} in which most application logic occur.

\subsection{Safelet}
\label{subsection:safelet}