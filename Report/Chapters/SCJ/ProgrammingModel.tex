\section{The Programming Model}
\label{section:programmingmodel}
The programming model of SCJ can be considered being much more strict than its predecessor, RTSJ, which has been criticised for being too complex and difficult to certify under rigid standards such as the DO-178B level A - something that was also mentioned in Section \ref{sub:brief_history_of_java_for_safety_critical_systems}. This simplification naturally poses restrictions on the developer in terms of how code must be structured within a SCJ application, but fortunately eases the whole certification process - a topic that will be described later in this Chapter. Furthermore, by forcing the developers to adhere to a general programming model as well as having special annotations for vendor-supplied third-party tools that are able to analyse correctness properties of the source code depending on the different use contexts, aids in simplifying the certification. This Section will describe the programming model in SCJ, starting with the so-called \textit{missions}, and the rationale behind some of the decisions and concepts.


\subsection{Missions}
As a requirement, every SCJ application will consist of at least one \code{mission}. Furthermore \code{missions} can be regarded as encapsulating a specific responsibility of the 

\subsection{Safelet}

\subsection{Handlers}