\chapter{Safety-Critical Java}
\label{chapter:scj}
Having introduced the area of real-time and safety-critical systems, this Chapter will introduce Safety-Critical Java. We begin with a general introduction to the programming model and the structuring of applications. Next, the concept of \textit{compliance levels} is introduced followed by a section about scheduling and concurrency. This section will also introduce some of the semantics and restrictions that applies for different SCJ applications. A significant aspect in SCJ is in its special treatment of memory management. This will be described as well as i/o handeling. Finally there will be a short overall description of the standard class library. 

Along the way way we will compare the various parts to the original RTSJ with the goal of emphasising some of the key differences as well as motivations behind why the work on SCJ was necessary (and similarly, these motivations also apply in general for why Ravenscar Java and Predictable Java were made). The chapter is primarily based on the SCJ specification draft, dated September 2012(cite til spec). We note that changes to the specification may occur after the finalisation of this report.