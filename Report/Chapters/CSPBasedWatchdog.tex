\chapter{A CSP based Watchdog in Safety-Critical Java}
\label{chapter:watchdogcsp}
During the first couple of weeks in which we were uncertain of the project direction, however knew that it would involve Safety-Critical Java, we wanted to spend some time in getting acquainted with the specification. After having read the first couple of chapters and being introduced to some of the most significant aspects, we decided to create a \textit{watchdog} application as a use case which would also include an appurtenant analysis. Be advised that the watchdog is \textit{not} related to CSP in any way whatsoever as this was done before learning about CSP. Moreover the design decisions of the application is solely based on learning experience and experimentation on how to execute SCJ applications on an FPGA that is configured with JOP. A mini report was written in conjunction with the application, documenting the entire process and analysis part. The report is included in Appendix \ref{appendix:WD} for reference and we suggest reading it before continuing. To summarise, the watchdog consists of three periodic event handlers performing the following duties:

\begin{description}
	\item[Pinger] This task periodically pings a number of registered modules and stores the response (if any) in a shared data structure.
	\item[Checker] This task checks all module responses in the shared data structure and sets a recovery flag in case a module has failed to reply.
	\item[Recovery] This task executes a recovery routine taking some user-defined appropriate action if the recovery flag it set. 
\end{description}

The tasks runs in sequence in the order: \textit{Pinger}, \textit{Checker} and \textit{Recovery} due to their inter-dependence. The FPGA, executing the watchdog, and the externally modules are connected through a bus and with communicating using the \iic protocol. Originally the functionality for \iic communication was integrated into the watchdog, however, this Section will describe how the watchdog was changed into using the developed CSP implementation instead. In total, three watchdog variants has been created and can be found in our GitHub repository~\cite{SW902e12:CSPinSCJ}

\begin{description}
	\item[Watchdog (Level 0)] Implementation conforming to compliance level 0 under a cyclic executive scheduler.
	\item[Watchdog (Level 1)] Implementation conforming to compliance level 1 under a fixed-priority scheduler. 
	\item[Watchdog (Level 1) CSP] Same as the above implementation, but instead of embedding \iic communication directly into the software, the developed CSP library is utilised enabling the possibility of using different communication interfaces depending on the hardware configuration.
\end{description}

\section{Modifications for Using CSP}
To incorporate the CSP library in the watchdog all communication classes and their respective packages are removed - i.e. \code{sw901e12.comm} and \code{sw901e12.comm.modules}. The field, \code{receivedResponseOnLastPing}, originally located on the now deleted \code{ModulePinger} is, however, transferred to the \code{Module} class itself. This class is also extended with fields for \code{CSPAddress}, \code{CSPPort} and \code{MACAddress}. In the mission, an array of modules are created with different properties depending on the execution context (board or simulator). The registration process of a module for the simulator and the board can be seen in Listing \ref{moduleboardsim}.

\lstinputlisting[label=moduleboardsim,caption=Module for the simulator and the board.]{Code/Implementation/moduleEx.java}

When run in the simulator, the \code{MACAddress} \code{CSPAddress} values for each module remains the same and are fetched from a \code{Config} class (its source and destination addresses are always the node is itself!). The \code{CSPPort} value, however, must vary for being able to distinguish between connections. Next the module is created with the mentioned parameters and added to the array of slave modules. The interface is set to Loopback with the same \code{MACAddress} and the node registered in the routing table. When run on the board, the \code{MACAddress}, \code{CSPAddress} must vary from the watchdog's information as the modules no longer reside within the same application. Now it is also necessary to supply a proper next hop MAC address to the requested interface based on the network topology (in the example, 0xFF). The implementation of the \code{handleAsyncEvent} is changed to use the CSP API. The changed \textit{Pinger} can be seen in Listing \ref{pingernew}.

\lstinputlisting[label=pingernew,caption=The Pinger handler.]{Code/Implementation/pinger.java}

In the old implementation of the \textit{Pinger} handler, the \code{ModulePinger} object for each slave was fetched and its \code{ping} method invoked as this would know the correct procedure of how to ping the particular slave device. This is no longer necessary when using CSP, as the protocol handles this internally using various implementations of MAC-layer protocols. The updated \textit{Checker} handler can be seen in Listing \ref{checkernew}.

\lstinputlisting[label=checkernew,caption=The Checker handler.]{Code/Implementation/checker.java}

The \textit{Checker}, has not changed much apart from going directly to the slave module in order to check the response flag. Finally the \textit{Recovery} handler can be seen in Listing \ref{recoverynew}.

\lstinputlisting[label=recoverynew,caption=The Recovery handler.]{Code/Implementation/recovery.java}

This handler did not require any changes and could be used as is.

\section{Determining WCET}

