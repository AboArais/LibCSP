\chapter{Conclusion}
\label{chapter:Conclusion}
During this project we worked on the basis of the following statement of the problem:
\begin{quotation}
	\textit{Can the network-layer delivery protocol, Cubesat Space Protocol, be designed and implemented using Safety-Critical Java, such that it can be used for applications that communicates with peripheral modules?}
\end{quotation}
The intention on working with this problem was to study the use of Safety-Critical Java and its specification, with focus on providing grounds for future work. The result was the finding of several areas of interest that may form the basis for future work.

We confirmed that CSP was able to be designed and implemented in SCJ, and we implemented a subset of CSP in SCJ as a level~1 application. An existing watchdog application in SCJ was created in a version based on the CSP implementation.

We found that the general programming model in SCJ is easily adaptable for experienced Java programmers. The choice of an event-based concurrency model maps well to the concept of tasks. We found that the scoped memory model and the requirement of explicit stating sizes of each memory area were the most challenging parts of SCJ. Tools for static analysis of memory allocation usage should be investigated for supporting the use of this memory model. With the new version of the DO-178B standard, garbage collection as a viable option should be investigated. For tasks that require synchronisation restrictions under level~1 compliance poses challenges on the development. The implementation of a blocking queue was not possible but instead had to rely on busy waiting. Execution time analysis of the watchdog, adapted for the CSP implementation, was complicated by this. Few tools supporting the development exist. Contributions on supporting compilation and debugging process is much needed in order for SCJ to be attractive. As a result of this study we state the following to sum up: 
\begin{quotation}
\centering
	\textit{Looks like Java, smells like Java - is not Java!}
\end{quotation}