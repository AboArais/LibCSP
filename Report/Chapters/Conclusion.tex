\chapter{Conclusion}
\label{chapter:Conclusion}
During this project we worked on the basis of the following problem statement:
\begin{quotation}
	\textit{How can the network-layer delivery protocol, Cubesat Space Protocol, be designed and implemented using Safety-Critical Java, such that it can be used for applications that communicates with peripheral modules?}
\end{quotation}
The intention on working with this problem was to study the use of Safety-Critical Java and its specification, with focus on providing grounds for future work. The result was the finding of several areas of interest that may form the basis for future work.

We were able to design and implement a subset of CSP in SCJ which conforms to compliance level~1. Also we created a simple client/server application using the protocol, but also changed the existing watchdog application into using the CSP implementation as well. Both of these use cases worked well and we were even able to find a few bugs in the implementation that were missed by the unit tests.

We found that the general programming model in SCJ easily adaptable for experienced Java programmers. The choice of an event-based concurrency model maps well to the concept of tasks and Java threads. We found the scoped memory model and the requirement of explicitly stating sizes of each memory area as being the most challenging parts of SCJ. Tools for static analysis of memory allocation usage should be investigated for supporting the use of this memory model. With the new version of the DO-178B standard, garbage collection as a viable option should be investigated. To ensure mutual exclusion through synchronisation under compliance level~1, more effort and creativity is imposed on the developer compared to how this concern is handled in standard edition Java. This is due to langauge limitations. The implementation of a blocking queue was not possible but instead we had to rely on busy waiting. Execution time analysis of the watchdog, adapted for the CSP implementation, was complicated by this. Few tools supporting the development exist. Contributions on supporting compilation and debugging process is much needed in order for SCJ to be attractive. As a result of this study we state the following to sum up: 
\begin{quotation}
\centering
	\textit{Looks like Java, smells like Java - is not Java!}
\end{quotation}