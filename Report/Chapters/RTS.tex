\chapter{Embedded, Real-Time and Safety-Critical Systems}
\label{chapter:rts}
This chapter introduces the concepts and definitions of embedded, real-time and safety-critical systems used throughout the report. The following is primarily based on the book, \textit{Real- Time Systems and Programming Languages}\cite{alan2001real}. Additional sources, are referred to as usual.

\section{Definitions and Background} % (fold)
\label{sec:definitions_and_background}
%In Chapter \ref{chapter:introduction} we presented a definition of real-time systems and briefly introduced the concepts of embedded and safety-critical systems. We will %here describe these in more detail.

\subsection{Embedded and Real-Time Systems} % (fold)
\label{sub:real_time_and_embedded_systems}
An embedded system can be thought of as a subsystem residing in a larger system such as the ABS braking system in a car. Upon hitting the brake pedal abruptly when driving on a slippery road, the ABS system will adjust the braking force such that the vehicle maintains its tractive contact with the road. We will thus use the following definition of an embedded system:

\begin{quotation}
``Any information processing activity or system which participates in a larger system and interact directly with the real world.''~\cite{alan2001real}
\end{quotation}

The example of an ABS braking system contains another important property which in fact also makes it a real-time system. When the driver brakes, the system must register this input and take appropriate action within reasonable time. Systems which also take timing considerations into account can be referred to as real-time systems. Moreover, a real-time system can be defined as:

\begin{quotation}
``Any information processing activity or system which has to respond to externally generated input stimuli within a finite and specified period.''~\cite{alan2001real}
\end{quotation}
%We defined a real-time system as a system in which correctness was not only dependent on the absence of failure in the generated output response, but also in the timing. 


Not all timing failures are, however, equally disastrous as the consequences may vary from simple usability annoyances to life threatening conditions as in the car example. We therefore distinguish between the following types of real-time systems:
\begin{description}
\item[Hard] Those real-time systems where responses must be generated within the specified deadline. Any miss of deadline results in failure. A pacemaker is one such example of a hard real-time system.
\item[Soft] Missing a deadline occasionally does not imply failure of the system, and it may continue to function correctly. An example of such a system could be a digital video-phone system in which delays in deliveries of network packets may result in quality degradation.
\item[Firm] As with soft real-time systems, deadlines can be missed occasionally, but late delivery in a response does not provide any benefit. A response may therefore be discarded in such case.
\end{description}

The different kinds of real-time systems are often associated with embedded systems. This can be as a component that monitors or controls operation of equipment. Because such applications will typically have timing constraints, embedded systems are typically also real-time systems (but these are by no means mutually inclusive).

\subsection{Safety-Critical Systems} % (fold)
\label{sub:safety_critical_systems}
% subsection safety_critical_systems (end)
Safety-critical systems can be categorised as hard real-time systems, as it is imperative that deadlines are satisfied. There are, however, different degrees of safety in the area of safety-critical software development. First, let us consider the definition of \textit{safety} in relation to software\cite{SCJSpec}:
\begin{description}
	\item[Safety:] A system property in which failure will not result in endangerment of human life or the environment.
	\item[Safety-critical system:] A system of high criticality with an extremely high assurance of the safety property.
\end{description}

It is important to distinguish between \textit{mission-critical} and safety-critical systems as these are very different from each other. In a mission-critical system, failures will only endanger the mission compared to safety-critical systems in which human lives are at stake. In this report we define a safety-critical system as:

\begin{quotation}
``A real-time system in which the consequence of an incorrect response or timing violation can directly result in significant human loss.''~\cite{alan2001real}
\end{quotation}

Just as we distinguish between different types of real-time systems, one can also distinguish between different criticality levels in the safety property, which is often done in the context of safety-critical systems. The level naturally depends on the specific application domain and the consequences anomalous behaviour may cast. We previously introduced the DO-178B standard, used by the FAA as a certification for safety-critical systems in the United States. This standard (and the European ED-12B equivalent) defines a total of five levels of software that is related to aircraft safety:
\begin{description}
	\item[Level A (Catastrophic)] Failure may directly result in, or contribute to, crash of the aircraft.  
	\item[Level B (Hazardous)] Failure may directly result in, or contribute to, the failure of a function that results in severe conditions for the aircraft. Failure of the function would reduce the capability of the aircraft or the ability of the crew to cope with converse operating conditions. As an example, this could be the weather radar during flight in an area with tropical storm.
	\item[Level C (Major)] Failures are still serious, but not as serious as level~B. Consequences may cause discomfort or increased crew workload. An example could be a system component that aids in many of the mental arithmetic tasks a pilot has. Failure of such a system puts this workload on the pilot instead.
	\item[Level D (Minor)] A failure may directly or contribute to a failure in a function that results in a minor failure. This failure does not reduce the safety significantly.
	\item[Level E (No Effect)] Failures does not result in any effect on the aircraft.
\end{description}

\subsection{Real-Time Tasks} % (fold)
\label{sub:real_time_tasks}
% subsection real_time_tasks (end)
In any real-time system, different computations are typically involved in order to generate a response. Such computations may be triggered by time or events. These computations are typically implemented as \textit{tasks}, and for reactive systems we distinguish between the following types of tasks:
\begin{description}
	\item[Periodic (time-triggered)] A task is periodic if it is executed periodically, with a defined release cycle time. An example could be a sensor monitor task that samples the sensor value every 50 ms.
	\item[Aperiodic (event-triggered)] A task is aperiodic if some event in the environment triggers the execution of the task, without any bound on how often this may be triggered. An example could be a computer mouse where the user generates click events. Such click events may arrive at any time without bound.
	\item[Sporadic (event-triggered)] A sporadic task is the same as an aperiodic task, except there is a minimum interarrival time. This could for instance be a coffee vending machine, in which the event generated to start brewing coffee will not be generated until after it has filled the current cup.
\end{description}

Rarely is it the case that only a single task exists. Instead a real-time (and safety-critical) system is typically made up of several tasks running concurrently or even in parallel. In the following we look at how this execution can achieved such that all temporal requirements are met.

% section definitions_and_background (end)

\section{Scheduling} % (fold)
\label{sec:scheduling}
In concurrent programs, i.e. programs containing multiple threads, it is not necessary to specify the exact ordering of task execution. The scheduler provides the means of sharing system resources with every participating thread in a way such that every thread meets its temporal requirements. For this to work the tasks rarely run until completion in a sequential manner one at a time. Instead the scheduler generates an internal \textit{schedule} based on the properties of each task, such that the tasks are interleaved with each other. If the program is correct, the output stays the same across all possible interleavings. Thus scheduling can be defined as:

\begin{quotation}
	\textit{The activity of restricting the non-determinism found in concurrent systems by ordering the execution of tasks, such that they meet their temporal requirements}
\end{quotation}

Several scheduling algorithms exist, including \textit{Cyclic Executive} (CE) and \textit{Fixed-Priority Scheduling} (FPS). Although many other algorithms exists such as \textit{Earliest Deadline First} (EDF) and \textit{Value-Based Scheduling} (VBS), attention will only be given to CE and FPS, as these are the ones employed in SCJ. Note that in the following, the notions of a \textit{task} and a \textit{thread} is used interchangeably.

\subsection{Cyclic Executive}
\label{subsec:cyclicExecutive}
When the number of tasks is fixed and periodic, it is possible\footnote{It is possible but often hard to create this schedule.} to create a schedule by hand such that when the schedule is repeated, every task meets its temporal requirements. Each task is basically broken down into fixed sized chunks of \textit{procedure calls} and represented in the table, which is known as the \textit{major cycle}. Typically the major cycle is split into time slots of a fixed duration, or simply \textit{minor cycles}, in a way such that each slot contains a small number of procedure calls. Clock interrupts every $x$ ms execute the procedures in a scheduled slot. Although the table can be difficult to construct, when tasks are long and many exist, the schedule requires no further correctness analysis, which is a benefit of this approach.. Table \ref{table:CEtaskset} and Figure \ref{img:scheduling_cyclic_example.pdf} shows an example of a task set with four tasks and the accompanying time-line.

\begin{table}[ht]
\centering
\begin{tabular}{c c c}
\hline
Task, N & Period, T & Computation time, C \\ [0.5ex]
\hline 
a & 50  & 20 \\
b & 50  & 18 \\
c & 100 & 9  \\
d & 100 & 6  \\ [1ex]
\hline
\end{tabular}
\caption{Task set.}
\label{table:CEtaskset}
\end{table}

\img{scheduling_cyclic_example.pdf}{0.9}{Time-line.}

The schedule repeats itself at time 100 (the major cycle) when the time slot with \textit{abc} runs again. It can therefore be concluded that the task set can be scheduled such that each task runs at its correct rate. 

\subsection{Fixed-Priority Scheduling}
\label{sub:fps}
Fixed-priority scheduling~(FPS), is a priority-based scheduling technique in which tasks prior to execution are assigned static priorities that are taking into account during scheduling decisions in a way such that a task with a high priority is preferred over a task with a low priority. As opposed to the previous Cyclic Executive scheme, the notion of a task (thread) is preserved in FPS, as tasks are not explicitly split up into procedure calls. Each task is always in one of the following states:

\begin{itemize}
	\item Runnable
	\item Suspended and is waiting for a timing event
	\item Suspended and is waiting for a non-timing event
\end{itemize}

The currently executing task is in a \textit{running} state, but can be \textit{suspended}, by e.g. the release of a higher priority task. In this case, the task waits for a timing event from the scheduler in order to continue. Tasks that are in a \textit{runnable} state, are ready to run, but are not yet scheduled by the scheduler. Finally tasks that are e.g. waiting for some condition, can be woken by a non-timing event, e.g. in the case of a producer-consumer pair. Figure \ref{img:task_states.pdf} depicts a state diagram of the possible states for a task, once it is created.
\img{task_states.pdf}{0.7}{State diagram for a task.}

\subsubsection{Preemption and Non-Preemption}
In many cases it makes sense to interrupt a lower priority task when a task with a higher priority is ready to run --- this is called \textit{preemption}. Sometimes, however, the lower priority task is allowed to continue to complete in case of another higher priority task being in a runnable state --- this is called \textit{non-preemption}. Often a preemptive approach is preferred over a non-preemptive as there are often reasons associated with assigning some tasks high priority, namely that these tasks should always be preferred to lower priority ones - in a non-preemptive solution this requirement is often violated letting the high priority tasks become less active. As two extreme alternatives, it is possible to delay the preemption of a task for some bounded time or simply just let the tasks decide when to release the CPU. These approaches are known as \textit{deferred preemption} and \textit{cooperative dispatching} respectively.

\subsubsection{Simple Task Model}
\label{subsub:taskmodel}
Analysis of programs is not an easy task, and when concurrency comes into play with multiple tasks that may use synchronisation mechanisms, this becomes a complex task. Therefore, the schedulability tests presented build upon \textit{the simple task model} that imposes restrictions in concurrent real-time systems. The simple task model makes the following assumptions of a concurrent real-time system:

\begin{itemize}
	\item The application is assumed to consist of a fixed set of tasks
	\item All tasks are periodic, with known periods
	\item The tasks are independent of each other
	\item All system overheads, context-switching times and so on are ignored (i.e. assumed to have zero cost)
	\item All tasks have deadlines equal to their periods (i.e. each task must complete before it is next released)
	\item All tasks have fixed worst-case execution times
	\item No task contains any internal suspension points (e.g. an internal delay statement or a blocking I/O request)
	\item All tasks execute on a single processor (CPU)
\end{itemize}

\subsubsection{Priority Assignment}
In FPS, the important question is how to assign the priorities amongst the different tasks. Naturally it would make sense to assign high priorities to tasks that are deemed important from a developer perspective. However, priorities are derived from temporal requirements rather than functional importance. Therefore, the following two methods are often used:

\begin{itemize}
	\item \textbf{Rate Monotonic:} Each task is given a unique priority based on its period - a short period equals a high priority. For example, two tasks, \textit{i} and \textit{j} with periods, $T_i$ and $T_j$ such that $T_i$ < $T_j$ yields the priority relationship, $P_i$ > $P_j$. Table \ref{table:FPSRateMon} shows an example of a task set with priorities set using rate monotonic. Note that 3 represents the highest priority and 1 is the lowest.

	\begin{table}[ht]
			\centering
			\begin{tabular}{c c c}
			\hline
			Task, N & Period, T & Priority, P \\ [0.5ex]
			\hline 
			a & 23 & 3 \\
			b & 41 & 2 \\
			c & 68 & 1 \\ [1ex]
			\hline
			\end{tabular}
			\caption{Example task set with assignment using rate monotonic.}
		\label{table:FPSRateMon}
	\end{table}

	\item \textbf{Deadline Monotonic:} Each task is given a unique priority based on its deadline - the task with the shortest deadline is assigned the highest priority. For example, two tasks \textit{i} and \textit{j} with deadlines, $D_i$ and $D_j$ such that $D_i$ < $D_j$ yields the priority relationship, $P_i$ > $P_j$. Table \ref{table:FPSDeadlineMon} shows an example of a task set with priorities set using deadline monotonic.

	\begin{table}[ht]
			\centering
			\begin{tabular}{c c c c}
			\hline
			Task, N & Period, T & Deadline, D & Priority, P \\ [0.5ex]
			\hline 
			a & 5  & 2  & 3 \\
			b & 12 & 4  & 2 \\
			d & 47 & 20 & 1 \\ [1ex]
			\hline
			\end{tabular}
			\caption{Example task set with assignment using deadline monotonic.}
		\label{table:FPSDeadlineMon}
	\end{table}

\end{itemize}

With rate monotonic, and in accordance with the previously mentioned simple task model, deadlines are implicit meaning that the period and deadline are assumed equal for each task ($D = T$). This assumption is lifted in the deadline monotonic scheme as deadlines are allowed to be less then the periods ($D <= T$). Additionally the requirement of having only periodic tasks can also be changed to support periodic and sporadic tasks. Doing so makes the deadline monotonic \textit{optimal} in the sense that if a task set can be scheduled under a preemptive fixed-priority scheduler using some fixed-priority assignment scheme while preserving this changed task model, then the task set can also be scheduled using the deadline monotonic scheme. Similarly, rate monotonic is optimal as a result of the following:
\begin{equation}
	D = T \Rightarrow Rm \equiv Dm 
\end{equation}

Figure \ref{img:fps_rate_mono_example.pdf} shows the time-line for the task set represented in Table \ref{table:FPSRateMon}. The computation times have been omitted for simplification. At the \textit{critical instant} all tasks are released, but only the task with the highest priority, namely task \textit{a}, is allowed to run. The remaining tasks are blocked until task \textit{a} finishes its release. Next task \textit{b} runs till completion after which task \textit{c} runs. Task \textit{c} is preempted once again, by the release of task \textit{a} the second time. After task \textit{a} has completed, task \textit{c} continues a little time before being preempted again by task \textit{b} this time. After this task has finished, task \textit{c} is finally able to complete.

\img{fps_rate_mono_example.pdf}{0.8}{Time-line for the task set in Tabel \ref{table:FPSRateMon}.}

% section scheduling (end)

\section{Testing Schedulability} % (fold)
\label{sec:schedulability_analysis}
A scheduling scheme consists not only of the algorithm for ordering the use of resources such as the CPU, but also provides a means for performing analysis in order to predict the worst-case behavior of the system under the particular scheduling algorithm. This is an important aspect of developing real-time and (in particular) safety-critical systems, as one must provide assurance that the temporal requirements of the tasks are satisfied. A schedulability test can have the following characteristics:
\begin{description}
	\item[Sufficient] A test is sufficient if it guarantees that all deadlines are always met when the test is satisfied.
	\item[Necessary] A test is necessary if failure to satisfy the test will result in a deadline miss at some point during execution.
\end{description}
A test that is both sufficient and necessary is considered \textbf{exact}. In the following we will examine two well known methods for determining schedulability of a concurrent system. Note that in the following we will apply the standard notation listed in Table \ref{table:ScheduleNotation} that is generally used in theory on real-time systems.
	\begin{table}[!h]
			\centering
			\rowcolors{1}{}{lightgray}
			\begin{tabular}{l l}
			\hline
			\textbf{Notation} & \textbf{Description}\\
			\hline 
			$B$ 	& Worst-case blocking time for the task (if applicable)  \\
			$C$ 	& Worst-case execution time (WCET) of the task \\
			$D$   & Deadline of the task \\
			$I$ 	& The interference time of the task \\
			$J$ 	& Release jitter of the task \\
			$N$ 	& Number of tasks in the system \\
			$P$ 	& Priority assigned to the task (if applicable) \\
			$R$ 	& Worst-case response time of the task \\
			$T$ 	& Minimum time between task releases (task period) \\
			$U$ 	& The utilisation of each task (equal to C/T) \\
			\hline
			\end{tabular}
			\caption{Standard notation\cite{alan2001real}.}
		\label{table:ScheduleNotation}
	\end{table}

\subsection{Utilisation-based Test} % (fold)
\label{sub:utilization_based_test}
This test can be considered the most simple test for FPS. It is important to note that while it is not exact (it is sufficient), it is simple to perform when applying the rate monotonic assignment scheme and thus makes the assumption of the simple task model.

The test is based on considering the utilisation of a task set. The schedulability is ensured --- i.e. all tasks in the task set will meet their deadlines if the following condition is true:

\begin{equation}
	\sum_{i=1}^N \biggl(\frac{C_i}{T_i}\biggr) \le N\bigl( 2^{1/N} - 1 \bigr)
\end{equation}

As stated the test, is not exact but only sufficient as there exists task sets for which the utilisation-based test will result in a negative outcome, but is actually schedulable. Therefore when the utilisation-based test fails, the task set \textit{may or may not} be schedulable. Variations and improvements for this test exists, however, these will not be discussed in this report. For additional information on the utilisation-based test and variations, we refer to the cited material.

% subsection the_simple_task_model (end)

\subsection{Response-Time Analysis (RTA)} % (fold)
\label{sub:response_time_analysis_}
Response-Time Analysis, or simply RTA, is an alternative test for FPS that is, in contrast to the utilisation-based test, exact, albeit more complex. Performing an RTA is divided into two overall steps:
\begin{enumerate}
\item Each task is analysed to compute its \textit{worst-case response time} (R)
\item The response time for each task is compared to its deadline
\end{enumerate}
The most difficult part of performing an RTA is naturally the analysis to find $R$ for all $N$ tasks. The analysis takes its starting point in the task with the highest priority. In a preemptive FPS system, $R$ for this task will always be equal to its own computation time, $C$. For the remaining tasks, the intuition is to calculate $R$ by analysing the interference from tasks with higher priority. Looking back to the possible states of a task in Section \ref{sub:fps}, this happens when a task is in the \textit{runnable} state, but cannot begin to execute because another task with a higher priority is in its \textit{running} state. For a a given task, $i$, the response time is found by calculating this interference and by adding its own computation time (under the assumptions of the simple task model):
\begin{equation}
	R_i = C_i + I_i
\end{equation}

In the worst case where all higher priority tasks are released at the same time (critical instant), the response time for $i$ can be expressed as the following:
\begin{equation}
	R_i = C_i + \sum_{j\in hp(i)} \ceil[\bigg]{\frac{R_i}{T_j}} \cdot C_j
\end{equation}

Because the value being calculated ($R_i$) is also present in the right hand side of the equation, this presents a fixed-point equation. The equation can be solved forming a recurrence relationship:
\begin{equation}
	\label{eq:recurrence}
	w^{n+1}_{i} = C_i + \sum_{j\in hp(i)} \ceil[\bigg]{\frac{w^n_i}{T_j}} \cdot C_j
\end{equation}

By solving the set of values \{$w^0_i, w^1_i, .. w^n_i, ..$\}, the solution to $R_i$ is found when $w^n_i = w^{n+1}_i$. In order to bootstrap the process, the value $C_i$ is used for $w^0_i$, under the safe assumption that $R_i \ge C_i$. In the process of solving the set of values, if $w^n_i$ becomes bigger than the period of task, then it will miss a deadline (recall that $D_i$ = $T_i$ in the simple task model). As a consequence, in order for a set of tasks to be deemed schedulable under the simple task model using the RTA schedulability test, the following property must be satisfied:
\begin{equation}
	\forall i \in N: R_i \le T_i
\end{equation}

\subsubsection{Example of Performing Response-Time Analysis}
Let us consider the use of RTA for the task set previously presented in Table \ref{table:FPSRateMon}. In Table \ref{table:FPSRateMonSecond}, each task have been provided with a computation time, $C$. In the following we will demonstrate the application of the response-time analysis method in order to verify the schedulability of the provided task set.
	\begin{table}
		\centering
		\begin{tabular}{c c c c}
		\hline
		Task & Period, T & Computation Time, C & Priority, P \\ [0.5ex]
		\hline 
		a & 23 & 4 &  3 \\
		b & 41 & 20 &  2 \\
		c & 68 & 10 &  1 \\ [1ex]
		\hline
		\end{tabular}
		\caption{Task set for analysis of schedulability.}
		\label{table:FPSRateMonSecond}
	\end{table}

As stated in Section \ref{sub:response_time_analysis_}, $R$ for the task with highest priority (in this case $a$) is set to its own computation time. Thus, we trivially find that $R_a = 4$. For task $b$ and $c$ we apply Equation \ref{eq:recurrence} to find $R_b$ and $R_c$:
\begin{eqnarray}
    w_{b}^0 &=& C_{b} = 20 \nonumber \\ 
    w_{b}^1 &=& 20 + \ceil[\bigg]{\frac{20}{23}} \cdot 4 = 24 \nonumber \\ 
    w_{b}^2 &=& 20 + \ceil[\bigg]{\frac{24}{23}} \cdot 4 = 28 \nonumber \\
    w_{b}^3 &=& 20 + \ceil[\bigg]{\frac{28}{23}} \cdot 4 = 28 \nonumber \\
    R_{b} &=& 28
\end{eqnarray}

\begin{eqnarray}
    w_{c}^0 &=& C_{c} = 10 \nonumber \\ 
    w_{c}^1 &=& 10 + \ceil[\bigg]{\frac{18}{23}} \cdot 4 + \ceil[\bigg]{\frac{18}{41}} \cdot 20 = 34 \nonumber \\ 
    w_{c}^2 &=& 10 + \ceil[\bigg]{\frac{34}{23}} \cdot 4 + \ceil[\bigg]{\frac{34}{41}} \cdot 20 = 38 \nonumber \\ 
    w_{c}^3 &=& 10 + \ceil[\bigg]{\frac{38}{23}} \cdot 4 + \ceil[\bigg]{\frac{38}{41}} \cdot 20 = 38 \nonumber \\ 
    R_{c} &=& 38
\end{eqnarray}

As a result we can see that the task set, under the simple task model, is indeed schedulable under FPS as it is the case that {$R_{a} < D_{a}$, $R_{b} < D_{b}$ and $R_{c} < D_{c}$}.

While the RTA test is exact, it is naturally very much restricted because of the assumptions made in the simple task model. Rarely is it the case that several concurrently running tasks are always independent of each other - synchronisation is often involved. For this reason there has been many extensions and variations of the described method to loosen or remove single restrictions in the simple task model. In Section \ref{sub:fps} we saw one example of this where the restriction of $D = T$ was lifted in the deadline monotonic assignment scheme. We refer to the cited material for additional information on how one may handle the removal of further restrictions (such as the inclusion of blocking time).

In addition to the restrictions in the task model, the schedulability test also depends on the important computation time, $C$, of a task being available. Up until now, we have simply made the assumption that this value was available. However, finding this value for a specific task is a complex process itself and presents many challenges that make up a complete research area on its own. In the following Section we will introduce the process and some of the methods for analysing a task in order to find the needed computation time.

% subsection response_time_analysis_ (end)

\subsection{Worst-Case Execution Time Analysis} % (fold)
\label{sub:worst_case_execution_time_analysis}
Worst-Case Execution Time (WCET) analysis of real-time programs is the process of determining an upper bound on the execution time on a specific piece of hardware\cite{Wilhelm:2008}. As stated previously, this is needed in order to show that a program satisfies its temporal requirements. Note that a thorough review of performing WCET analysis is beyond the scope of this report - for more details we refer to the cited material.

Methods (or direct tools) for determining the WCET of an application are evaluated on two criteria. \textit{Safety} describes whether the method produces only estimates or actual bounds. \textit{Precision} describes how close the bounds or estimates are to the exact values. Methods are generally classified as either a \textit{static} analysis technique or \textit{end-to-end measurement}, both of which have varying safety and precision properties. In the following, we will briefly provide an introduction on these to give an intuition behind the two overall approaches.

\subsubsection{End-to-End Measurement}
End-to-end measurement of code is, as the name may imply, based on doing manual measurement of executed code, that is, a given task is executed on the desired hardware platform, or in a simulator while being measured. The measurement can be done using metrics such as time or CPU cycles. The technique presents several challenges such as choosing appropriate input for a task that results in worst-case behaviour, and varying execution times on the same input e.g. due to caching.
In Figure \ref{img:wcet_terms.pdf} the problem of WCET analysis is illustrated along with an example of the type of results an end-to-end measurement may provide.

\img{wcet_terms.pdf}{0.8}{Basic notation in WCET analysis and illustration of the problem - adapted and modified from the article, \textit{The worst-case execution-time problem-overview of methods and survey of tools}\cite{Wilhelm:2008}.}

The dashed curve represents the execution time of all executions, where the solid curve represents measured execution times over a number of trials. As it can be seen, measurement provides only estimates for what is called \textit{minimal observed} and \textit{maximal observed execution times}. This technique is therefore not safe, as it typically overestimates the lower bound, and, more importantly, underestimates the upper bound. In terms of handling the underestimation of the upper bound, it is common to add some percentage to the calculated WCET. 

\subsubsection{Static Analysis}
This class contains the methods that do not involve execution of code on real hardware or through simulation. While several methods exists, one example of uses \textit{Control-Flow Analysis} (CFA) that can also be used in combination with \textit{Data-Flow Analysis} (DFA). CFA is based on examining the execution paths of a program and constructing a \textit{Control-Flow Graph} (CFG). Using this CFG, a tool may analyse execution paths while counting information such as instructions for functions.
Static approaches that uses e.g. CFA provides a safe approximation of the bound. This is also illustrated in Figure \ref{img:wcet_terms.pdf} as the upper bound provided that is an over estimate of the actual possible WCET. 
%However, the use of static techniques presents some issues. First, programs can be complex, and building the CFG can be a problematic process. Usually tools rely on the programmer providing hints for the tool, such as explicit bounds on loops to reduce the complexity of the static analysis. Second, the hardware has become very complex which makes it more difficult for a static technique to make an appropriate abstraction of the underlying hardware.

\subsubsection{Hybrid Techniques}
As a result of the different issues found in both measurement and static techniques, one can also apply a hybrid based approach. Depending on the program, tasks can be decomposed into smaller functions that can separately be analysed using an appropriate technique with all the results being aggregated in the end.


% section summary (end)
