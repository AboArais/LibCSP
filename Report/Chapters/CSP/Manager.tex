\subsection{Public API}
How application developers use the CSP implementation will be through a public interface that only exposes the needed operations. 

The first we consider is the classes that require visibility for the developer. These are the \code{Socket}, \code{Connection} and \code{Packet} classes. Rather than directly exposing these classes, which also rely on having additional operations for internal use, we provide simplified visibility through what could be compared to the use of the proxy pattern\cite{Gamma:1995:DPE:186897}. Having an indirection between the developer and these classes has two primary advantages. First, all methods and fields on the classes that are only to be used internally are no longer visible. Second and most importantly, we prevent further instantiations of objects, as only the ones from the resource pool should be used. The disadvantage for this approach is that there will be an increased footprint in having additional classes that acts as the proxies for the underlying ones. However, as this only holds for the previously stated three classes, the tradeoff is minimal --- in addition this can be done using interfaces in Java.

The second part of the public API we consider is the general access to the library. Functionality to initialise the library and get \code{Socket}, \code{Connection} and \code{Packet} objects is necessary. However, we do not want to expose the resource pool, as this would put additional responsibility on developers to e.g. put objects back into the pool (forgetting would result in memory leaks). This was seen in the example C application, using the original C implementation, where packet structs must be explicitly freed using their buffer API. For this we provide the class \code{CSPManager}. This single entity will provide operations for parameterising and initialising the library as well as providing an interface for the resource pool. \code{CSPManager} will be a combination of an architectural facade and a factory\cite{Gamma:1995:DPE:186897}. To summarise, Figure \ref{img:manager.pdf} illustrates the concept of the the indirections and the \code{CSPManager} class.
\img{manager.pdf}{0.8}{Indirections hiding the inner implementation for client applications.}