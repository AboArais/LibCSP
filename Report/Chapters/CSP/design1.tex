\section{Design}
This section will elaborate on the most important design decisions. The considerations and decisions are based on the review of the C implementation in conjunction with the SCJ specification.

\subsection{Ports, Sockets, Connections and Packets}
We begin with some of the basic concepts in CSP, and how the design of sockets must accommodate the available features. A socket entity represents an endpoint. In addition to the socket, a connection can exist between endpoints. We emphasize that this concept of a connection is not related to that of connection-less and connection-oriented sockets.  On the server side, an application will bind a socket to a specified port, and all subsequent communication on that port will end up with a new connection. Therefore, individual packets are also read and sent through connections. A connection is uniquely identified on a host by the 4-tuple:
\begin{quotation}
	\emph{(destination address, destination port, source address, source port)}
\end{quotation}

Consider the network example illustrated in Figure \ref{img:network_terminology_connections.pdf}. Here, the \textit{NanoCam} has an open socket on port 28. Both the \textit{NanoMind} and \textit{Mission Control} communicates to the \textit{NanoCam} using this open end-point (each using a randomly selected outgoing port - 49 and 52 respectively). 
\img{network_terminology_connections.pdf}{0.60}{Example scenario with active connections using CSP. The tuples like (x,~y) and (x,~y,~z,~w) represents (source~address,~source port) and (destination~address,~destination~port,~source~address,~source~port) respectively.}

The relationship between ports, sockets, connections and packets is illustrated in Figure \ref{img:overallflow.pdf}. Note that while source and destination ports occupies 6 bits each in the header, only values $0-47$ are available for incoming connections (of which port 0-8 is reserved for ICMP-like requests). The remaining ports up to 63 are used for outgoing connections.
\img{overallflow.pdf}{0.9}{Relationship between a socket bound to a port and with a list of connections each having its own list of packets.}

As a result of this, we have the following classes to support these concepts:
\begin{description}
	\item[Port] The port class is used to associate a port with a socket. The routing logic will be able to use objects of the port class to determine if a given port is open upon receiving new packets and retrieve the associated socket.
	\item[Socket] The socket class is used to define an end-point that can be bound to a specific port that an application should listen on. Furthermore this will provide the access to accept new incoming connections.
	\item[Connection] The connection class is used to represent incoming and outgoing activity between two end-points. Packets are sent and read between end-points through this entity. A connection is always associated with a socket on the initial receiving side.
	\item[Packet] The packet class represents the CSP entity packet that is sent through connections. This contains the header and payload.
\end{description}
