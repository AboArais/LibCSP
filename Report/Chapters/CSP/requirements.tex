\section{Requirements}
Although the current version of CSP supports many features, we choose to implement a subset of these. We focus on parts of CSP in each layer of the stack, that supports further implementation of the remaining features. The primary goal is to implement a minimised version of the full protocol stack. The requirements are delimited to the following features:

\begin{itemize}
	\item Thread-safe socket API
	\item Unreliable packet transmission (UDP)
	\item Broadcast traffic
	\item Loopback interface
	\item \iic interface
\end{itemize}

The selected features are primarily chosen as these constitute substantial parts in each layer. Additional features can then be incorporated (e.g. RDP is implemented using UDP or ICMP in a simple application using the socket API).
Note that when we refer to thread-safety, this effectively means safe execution by multiple event handlers in the course of SCJ.

We introduced JOP as the underlying JVM with its SCJ implementation as the platform for this project. In addition to the specified features we specify the following non-functional requirements:

\begin{itemize}
	\item Conformance with compliance level~1 - working with this level compared to a level~0 implementation, will provide more depth and increased exploration of the specification in the study.
	\item The implementation must be within the boundaries of the SCJ specification. The use of hardware objects for external device support an exception. Any further deviations must be explicitly stated with grounds.
\end{itemize}

No temporal requirements will be specified for this, as these are relevant for real-time applications using the protocol.