\subsection{MAC-layer Protocols} % (fold)
\label{sub:mac_layer_protocols}
In order to achieve modularity and simplify the implementation of further MAC-layer protocols, the logic for reading a frame in the protocols are not seperately implemented as aperiodic event handlers. Instead the interface \code{IMACProtocol} is created which defines the three methods \code{initialize(int macAddress)}, \code{transmitPacket(PacketCore packet)} and \code{receiveFrame()}. All MAC-layer protocol implementations must implement this interface. For \iic, an extract illustrating this approach is seen in Listing \ref{iicprotocol}. The loopback is implemented by simply placing packets back to the routing handler packet queue upon transmit.
\lstinputlisting[label=iicprotocol,caption=Extract of the \iic protocol implementation.]
{Code/Implementation/interfacei2c.java}

A common aperiodic event handler implementation can then be made that can be used for any protocol. The class \code{ISRHandler} is the implementation for the aperiodic event handlers, and is seen in Listing \ref{isrhandler}.
\lstinputlisting[label=isrhandler,caption=A common aperiodic event handler for all MAC-layer protocols.]
{Code/Implementation/interfaceisr.java}

On JOP the handler object for a protocol must explicitly be bound to a corresponding interrupt fired by the microcontroller for the hardware interface during the \textit{initialization} phase.

% subsection mac_layer_protocols (end)