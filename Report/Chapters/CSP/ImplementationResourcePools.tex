\subsection{Resource Pools} % (fold)
\label{sub:resource_pools}
The resource pools are managed in a \code{ResourcePool} object. The pool uses three separate queues for storing unused \code{SocketCore}, \code{ConnectionCore} and \code{PacketCore} objects. During the \textit{initialization} phase the \code{CSPManager} will initialise the pools with a fixed amount of objects. As illustrated in Listing \ref{poolinit}, this is done simply by enqueuing new instantiations of the objects. All are therefore allocated in mission memory and none of these will end up referencing to other objects in a shorter lived scope. 
\lstinputlisting[language=Java,label=poolinit,caption=Initialising the pool for packets.]
{Code/Implementation/resourcepoolinit.java}

Subsequent retrieval and return of objects in a pool is done through the enqueue and dequeue operations on the particular queue. For the packet queue, this can be seen in Listing \ref{pooluse}. For all internal use of the pools, the timeout is always set to only perform a single dequeue attempt. No internal parts of the implementation thus performs any waiting if no resources are available, as this would make the execution time less transparent for users.
\lstinputlisting[language=Java,label=pooluse,caption=Using the resource pools.]
{Code/Implementation/pooluse.java}

Upon retrieving an object from the pool the field must naturally be assigned values. Similarly, when the object is no longer of use it must be returned to its pool. In this case its fields should also be cleared before returning it. For this reason all classes for objects kept in pools, will implement an \code{IDispose} interface that defines the \code{dispose} method. This simplifies this process of returning an object. Consider a \code{ConnectionCore} object that should be returned. Aside from clearings its own fields and returning itself, it also has its own queue of packets. Therefore any remaining packets that was not read should also be returned to their respective pool. Simply invoking the \code{dispose} method on any pool object allows easy separation of such operations. The implementation of \code{dispose} for \code{ConnectionCore} is seen in Listing \ref{dispose}. Declaring the type parameter \code{T}, in the generic \code{Queue}, as a bounded type argument that implements \code{IDispose} also simplifies the reset of a queue.
\lstinputlisting[language=Java,label=dispose,caption=Implementing the IDispose interface in ConnectionCore.]
{Code/Implementation/connectiondispose.java}

% subsection resource_pools (end)