\section{Design}
This Section will elaborate on the most important design decisions. The considerations and decisions are based the review the C implementation in conjunction with the SCJ specification.

\subsection{Routing}
In CSP, and any other network for that matter, \textit{routing} is a central task that revolves around choosing network trafficking paths along a network. To obtain an overview of the interplay between routing and the remaining system, we abstractly outline the flow with respect to our implementation requirements. Figure \ref{img:stack_flow.pdf} shows the protocol stack with the major components inside. As illustrated in the Figure, frames arrive and depart from the \textit{drivers} layer through a combination of VHDL and Java code. At the \textit{MAC-layer protocols} layer, for a frame arriving from some particular interface (\iic or Loopback), the frame exterior is removed and the packet placed in a packet-collection residing in the \textit{routing core} layer. This collection stores all packets to be processed by the routing task. Similarly, packets arriving from user applications caused by "send" invocations at the \textit{Transport extensions} layer also end up in this data structure. Regularly the routing task extracts a packet from the collection and sends it to the upper or lower layer depending on the header information. If the destination address of the packet matches the node, a port table is  


% Routing entiten
% Node

\img{stack_flow.pdf}{0.7}{Abstract flow.}