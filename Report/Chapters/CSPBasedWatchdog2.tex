\section{Response-Time Analysis}
To create a RTA for the new watchdog application, the requirement of task independence from the simple task model in Section \ref{subsub:taskmodel} will have to be loosened. Moreover, some of the tasks share critical sections throughout execution which can infer blocking times that must be included. To extend the response-time analysis to include blocking times, the execution times of the involved shared critical sections (only synchronised methods now) must be found. These critical sections and their found execution times, found using the static analysis tool available for JOP, are:

\begin{itemize}
	\item $ sw901e12.csp.core.ConnectionCore.processPacket \approx 0.0247\ ms$
	\item $ sw901e12.csp.core.SocketCore.processConnection \approx 0.0159\ ms$
	\item $ sw901e12.csp.util.Queue.enqueue \approx 0.0125\ ms$
	\item $ sw901e12.csp.util.Queue.dequeue \approx 0.0130\ ms$
\end{itemize}

Equation \ref{eq:block} finds the maximum blocking time for a task $i$ (in this case $K=4$). $usage$ is a 0/1 function and will return $1$ if resource $k$ is used by $i$ and there exists at least one task with a higher priority that also uses $k$ and at least one lower priority task that similarly uses $k$.

\begin{equation}
\label{eq:block}
     B_{i} = \overset{K}{\underset{k=1}{max}}\ usage(k, i)C(k)
\end{equation}

The response-time of a task, taking blocking into account can be seen in Equation~\ref{eq:rtablock}\cite{alan2001real}.

\begin{equation}
\label{eq:rtablock}
     R_{i} = C_{i} + B_{i} + \sum\limits_{j\in hp(i)} \ceil{\frac{R_{i}}{T_{j}}} \cdot C_{j}
\end{equation}

The blocking time of the aperiodic event handler for incoming \iic data, \textit{APEH}, is seen in Equation \ref{eq:isrblock}. This task has the highest priority and uses the critical sections, \code{enqueue} and \code{dequeue}, which are also used by lower priority tasks (\textit{Routing} and \textit{Pinger}).

\begin{eqnarray}
\label{eq:isrblock}
     B_{apeh} &=& max(0 \cdot 0.0247, 0 \cdot 0.0159, 1 \cdot 0.0125, 1 \cdot 0.0130) \nonumber \\
     B_{apeh} &=& 0.0130
\end{eqnarray}

The response time of the task can be seen in Equation \ref{eq:isr}.

\begin{eqnarray}
\label{eq:isr}
      R_{apeh} &=& C_{apeh} + B_{apeh} \nonumber \\
      R_{apeh} &=& 0.1342 + 0.0130 = 0.1472
\end{eqnarray}

The \textit{routing} task is a bit more complicated as it can be interrupted by the aperiodic \textit{APEH} task which have no period defined. First, the possible blocking time must be calculated as seen in Equation \ref{eq:routingblock}.

\begin{eqnarray}
\label{eq:routingblock}
     B_{routing} &=& max(0 \cdot 0.0247, 0 \cdot 0.0159, 1 \cdot 0.0125, 1 \cdot 0.0130) \nonumber \\
     B_{routing} &=& 0.0130
\end{eqnarray}

To find the response time, Equation \ref{eq:rtablock} cannot be directly applied because \textit{APEH} has no period. However, the expression appearing in the equation; $\ceil{\frac{R_{i}}{T_{j}}}$, states the number of releases of a higher priority task $j$ within the time interval $[0-R_{i}]$. For the \textit{routing} task we know that it can only be interrupted once by the \textit{APEH} task - i.e. in case the \textit{Pinger} task sends a packet but is preempted by the \textit{Routing} task, before receiving a response. In this case, the \textit{Routing} task can in the worst-case be preempted a single time if a response packet arrives and is being processed by the \textit{APEH} task. The response time thus becomes as seen in Equation \ref{eq:routing}:

\begin{eqnarray}
\label{eq:routing}
      R_{routing} &=& C_{routing} + (1 \cdot C_{apeh}) + B_{routing} \nonumber \\
      R_{routing} &=& 1.3623 + 0.1342 + 0.0130 = 1.5095
\end{eqnarray}

The blocking time for the \textit{Pinger} task can be seen in Equation \ref{eq:pingerblock}.

\begin{eqnarray}
\label{eq:pingerblock}
     B_{routing} &=& max(0 \cdot 0.0247, 0 \cdot 0.0159, 0 \cdot 0.0125, 0 \cdot 0.0130) \nonumber \\
     B_{routing} &=& 0
\end{eqnarray}

From Equation \ref{eq:rtablock}, the response-time becomes as seen in Equation \ref{eq:pinger}. In the worst case, the \textit{APEH} can block the \textit{Pinger} 10 times in case of receiving reponse from all 10 modules.

\begin{eqnarray}
\label{eq:pinger}
    w_{pinger}^0 &=& C_{pinger} = 163.75 \nonumber \\ 
    w_{pinger}^1 &=& 163.75 + \ceil{\frac{163.75}{5}} \cdot 1.3623 + (10 \cdot 0.1342) = 210.0479 \nonumber \\ 
    w_{pinger}^2 &=& 163.75 + \ceil{\frac{210.0479}{5}} \cdot 1.3623 + (10 \cdot 0.1342) = 223.6709 \nonumber \\
    w_{pinger}^3 &=& 163.75 + \ceil{\frac{223.6709}{5}} \cdot 1.3623 + (10 \cdot 0.1342) = 226.3955 \nonumber \\
    w_{pinger}^4 &=& 163.75 + \ceil{\frac{226.3955}{5}} \cdot 1.3623 + (10 \cdot 0.1342) = 227.7570 \nonumber \\
    w_{pinger}^5 &=& 163.75 + \ceil{\frac{227.7570}{5}} \cdot 1.3623 + (10 \cdot 0.1342) = 227.7570 \nonumber \\
    R_{pinger} &=& 227.7570
\end{eqnarray}

The remaining tasks, the \textit{Checker} and \textit{Recovery} tasks are not complicated by any possible blocking. Their only dependency is still solved through the use of the priority assignments. Because these will run after the \textit{pinger} is finished exectuing, this also means the \textit{pimnger} is done pinging modules. We make the assumption that no other data will arrive, thus the \textit{APEH} task will not be fired during these. The following is the response times found for the \textit{Checker} and \textit{Recovery} tasks.
\begin{eqnarray}
\label{eq:checker}
    w_{checker}^0 &=& C_{checker} = 0.0648 \nonumber \\ 
    w_{checker}^1 &=& 0.0648 + \ceil{\frac{0.0648}{5}} \cdot 1.3623 + \ceil{\frac{0.0648}{500}} \cdot 163.75 = 165.1771 \nonumber \\ 
    w_{checker}^2 &=& 0.0648 + \ceil{\frac{165.1771}{5}} \cdot 1.3623 + \ceil{\frac{165.1771}{500}} \cdot 163.75 = 210.133 \nonumber \\
    w_{checker}^3 &=& 0.0648 + \ceil{\frac{210.133}{5}} \cdot 1.3623 + \ceil{\frac{210.133}{500}} \cdot 163.75 = 222.3937 \nonumber \\
    w_{checker}^4 &=& 0.0648 + \ceil{\frac{222.3937}{5}} \cdot 1.3623 + \ceil{\frac{222.3937}{500}} \cdot 163.75 = 226.4806 \nonumber \\
    w_{checker}^5 &=& 0.0648 + \ceil{\frac{226.4806}{5}} \cdot 1.3623 + \ceil{\frac{226.4806}{500}} \cdot 163.75 = 226.4806 \nonumber \\
    R_{checker} &=& 226.4806
\end{eqnarray}

\begin{eqnarray}
\label{eq:recovery}
    w_{checker}^0 &=& C_{recovery} = 0.53895 \nonumber \\ 
    w_{checker}^1 &=& 0.53895 + \ceil{\frac{0.53895}{5}} \cdot 1.3623 + \ceil{\frac{0.53895}{500}} \cdot 163.75 + \ceil{\frac{0.53895}{500}} \cdot 0.0648  = 165.716 \nonumber \\ 
    w_{checker}^2 &=& 0.53895 + \ceil{\frac{165.716}{5}} \cdot 1.3623 + \ceil{\frac{165.716}{500}} \cdot 163.75 + \ceil{\frac{165.716}{500}} \cdot 0.0648  = 210.671 \nonumber \\ 
    w_{checker}^3 &=& 0.53895 + \ceil{\frac{210.671}{5}} \cdot 1.3623 + \ceil{\frac{210.671}{500}} \cdot 163.75 + \ceil{\frac{210.671}{500}} \cdot 0.0648  = 222.932 \nonumber \\ 
    w_{checker}^4 &=& 0.53895 + \ceil{\frac{222.932}{5}} \cdot 1.3623 + \ceil{\frac{222.932}{500}} \cdot 163.75 + \ceil{\frac{222.932}{500}} \cdot 0.0648  = 225.657 \nonumber \\ 
    w_{checker}^5 &=& 0.53895 + \ceil{\frac{225.657}{5}} \cdot 1.3623 + \ceil{\frac{225.657}{500}} \cdot 163.75 + \ceil{\frac{225.657}{500}} \cdot 0.0648  = 227.019 \nonumber \\ 
    w_{checker}^6 &=& 0.53895 + \ceil{\frac{227.019}{5}} \cdot 1.3623 + \ceil{\frac{227.019}{500}} \cdot 163.75 + \ceil{\frac{227.019}{500}} \cdot 0.0648  = 227.019 \nonumber \\ 
    R_{checker} &=& 227.019
\end{eqnarray}

As can be seen the following condition holds and the watchdog is still schedulable:
$R_{router} < D_{router}$, $R_{pinger} < D_{pinger}$, $R_{checker} < D_{checker}$, and $R_{recovery} < D_{recovery}$

