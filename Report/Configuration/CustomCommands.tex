\usepackage{graphicx}
\usepackage{float}
\usepackage{subfigure}
\usepackage{listings}
\usepackage{color}
\usepackage{inconsolata}
\usepackage{lastpage}
\usepackage{caption}
\usepackage{lmodern}
\usepackage{tikz}
\usetikzlibrary{shapes.callouts,shadows, calc}
\usepackage{courier}
\usepackage{mathtools}
\DeclarePairedDelimiter{\ceil}{\lceil}{\rceil}

\definecolor{lightgray}{gray}{0.9}

\graphicspath{{./Images/}}

% Code font (Looks like consolas but more nice!)
\newcommand{\code}[1] {{\fontfamily{fi4}\selectfont#1}}

% I2C cmd
\newcommand{\iic}{I$^{\textrm{2}}$C }

\newcommand{\img}[3]{
	\begin{figure}[!htp]
			\centering
			\includegraphics[width=#2\textwidth]{#1}
			\caption{#3}
			\label{img:#1}
	\end{figure} 
}

\newcommand{\imgRotated}[4]{
  \begin{figure}[!htp]
      \centering
      \includegraphics[width=#2\textwidth, angle=#4]{#1}
      \caption{#3}
      \label{img:#1}
  \end{figure} 
}

\newcommand{\imghere}[3]{
	\begin{figure}[H] 
			\centering
			\includegraphics[width=#2\textwidth]{#1}
			\caption{#3}
			\label{img:#1}
	\end{figure} 
}

\definecolor{dkgreen}{rgb}{0,0.6,0}
\definecolor{gray}{rgb}{0.5,0.5,0.5}
\definecolor{mauve}{rgb}{0.58,0,0.82}

\tikzset{note/.style={rectangle callout, rounded corners,fill=gray!20,drop shadow,font=\footnotesize}}    

\newcommand{\tikzmark}[1]{\tikz[overlay,remember picture] \node (#1) {};}    

\newcounter{image}
\setcounter{image}{1}

\makeatletter
\newenvironment{btHighlight}[1][]
{\begingroup\tikzset{bt@Highlight@par/.style={#1}}\begin{lrbox}{\@tempboxa}}
{\end{lrbox}\bt@HL@box[bt@Highlight@par]{\@tempboxa}\endgroup}

\newcommand\btHL[1][]{%
  \begin{btHighlight}[#1]\bgroup\aftergroup\bt@HL@endenv%
}
\def\bt@HL@endenv{%
  \end{btHighlight}%   
  \egroup
}
\newcommand{\bt@HL@box}[2][]{%
  \tikz[#1]{%
    \pgfpathrectangle{\pgfpoint{0pt}{0pt}}{\pgfpoint{\wd #2}{\ht #2}}%
    \pgfusepath{use as bounding box}%
    \node[anchor=base west,rounded corners, fill=green!30,outer sep=0pt,inner xsep=0.2em, inner ysep=0.1em,  #1](a\theimage){\usebox{#2}};
  }%
   %\tikzmark{a\theimage} <= can be used, but it leads to a spacing problem
   % the best approach is to name the previous node with (a\theimage)
 \stepcounter{image}
}
\makeatother

\lstdefinelanguage{own}
{
morecomment = [l]{//},
 morecomment = [l]{///},
 morecomment = [s]{/*}{*/},
 morestring=[b]",
 sensitive = true,
 morekeywords = {abstract,  event,  new,  struct,
  as,  explicit,  null,  switch,
  base,  extern,  object,  this,
  bool,  false,  operator,  throw,
  break,  finally,  out,  true,
  byte,  fixed,  override,  try,
  case,  float,  params,  typeof,
  catch,  for,  private,  uint,
  char,  foreach,  protected,  ulong,
  checked,  goto,  public,  unchecked,
  class,  if,  readonly,  unsafe,
  const,  implicit,  ref,  ushort,
  continue,  in,  return,  using,
  decimal,  int,  sbyte,  virtual,
  default,  interface,  sealed,  volatile,
  delegate,  internal,  short,  void,
  do,  is,  sizeof,  while,
  double,  lock,  stackalloc,  
  else,  long,  static,  
  enum,  namespace,  string}
}
 

\lstset{
      language=own,
         basicstyle=\footnotesize\ttfamily,
         numbers=left,               
         numberstyle=\tiny,
         numbersep=5pt, 
         tabsize=4, 
         extendedchars=true,
         breaklines=true, 
         keywordstyle=\color{red},
        frame=b,         
 %        keywordstyle=[1]\textbf,    % Stil der Keywords
 %        keywordstyle=[2]\textbf,    %
 %        keywordstyle=[3]\textbf,    %
 %        keywordstyle=[4]\textbf,   \sqrt{\sqrt{}} %
         stringstyle=\color{black}\ttfamily, % Farbe der String
         showspaces=false,           % Leerzeichen anzeigen ?
         showtabs=false,             % Tabs anzeigen ?
         xleftmargin=17pt,
         framexleftmargin=17pt,
         framexrightmargin=5pt,
         framexbottommargin=4pt,
         %backgroundcolor=\color{lightgray},
         showstringspaces=false,      % Leerzeichen in Strings anzeigen ?        
         moredelim=**[is][\btHL]{`}{`},
 }
    %\DeclareCaptionFont{blue}{\color{blue}} 

  %\captionsetup[lstlisting]{singlelinecheck=false, labelfont={blue}, textfont={blue}}
\DeclareCaptionFont{white}{\color{white}}
\DeclareCaptionFormat{listing}{\colorbox[cmyk]{0.43, 0.35, 0.35,0.01}{\parbox{\textwidth}{\hspace{15pt}#1#2#3}}}
\captionsetup[lstlisting]{format=listing,labelfont=white,textfont=white, singlelinecheck=false, margin=0pt, font={bf,footnotesize}}

