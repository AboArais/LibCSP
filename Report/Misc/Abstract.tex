Recently Java has been targeted the development of safety-critical systems with the new standard, JSR-302 \textit{Safety-Critical Java} (SCJ), to facilitate the development of systems, certifiable with standards such as \textit{DO-178B Level~A}. The SCJ specification is still in draft, and lacks implementations and use cases in order to be thoroughly evaluated. 

This report presents a study of SCJ and the application of this for two use cases.
We examine the September 2012 draft version of the SCJ specification, and an implementation of SCJ on the \textit{Java Optimized Processor} (JOP). We take the specification and its implementation on JOP and apply this in the development of safety-critical software in collaboration with GomSpace that specialises in cubesat and nano-satellite platforms. We develop an SCJ version of their \textit{Cubesat Space Protocol} (CSP), a network-layer delivery protocol for cubesats. Furthermore we take this implementation and develop a watchdog module in SCJ that utilises the CSP protocol to monitor other modules in a safety-critical environment and analyse the applicability of SCJ from a theoretical point of view with regard to theory in real-time software such as response-time analysis. The result of this study is an analysis and findings on the applicability of SCJ in its current state for the development of safety-critical systems with emphasis on transitioning ones Java experience to SCJ.