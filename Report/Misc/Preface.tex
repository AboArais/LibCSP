\newpage
\thispagestyle{empty}
\mbox{}

\chapter*{Preface}
This project has been developed by two Software Engineering speciality students from Aalborg University in fall 2012. The report documents the use of the \textit{Safety-Critical-Java} (SCJ) specification draft in terms of developing use cases for it.

\vspace{4mm}
\noindent The report consists of ten chapters, excluding the appendix. The first chapter introduces the areas of real-time and safety-critical systems and provides a brief overview of the history and currently hot research topics in relation to SCJ. The chapter will end with a problem statement that sets the scope of the project. The next chapter introduces the concepts and terminology of real-time systems followed by a chapter covering some of the relevant aspects of SCJ. Next, the \textit{Java Optimized Processor} (JOP), is covered, that provides the most significant implementation of SCJ at the moment.

With the theoretical grounds described, the report shifts into a more hands-on approach in terms of covering the development of a library as a use case, more specifically the \textit{CubeSat Space Protocol} (CSP), under SCJ specification restrictions. This will be in the shape of two chapters, one for explaining the CSP protocol and one for the development. The next chapter will cover a watchdog application that uses the developed protocol.

Finally the project is concluded in a three closing chapters covering reflection, future works and conclusion. 

\vspace{4mm}
\noindent Source material referenced in this report will be notated with the initial letter of the surname of the author(s), followed by the year of publication. For example, is a web page written by Egham published in 2011. The source refers to an entry in the bibliography list, where details regarding the source can be found.

\vspace{5mm}
	\begin{flushright}
\emph{Enjoy reading!} Group sw901e12
	\end{flushright}

\newpage
